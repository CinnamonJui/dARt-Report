\documentclass[12pt, a4paper]{report}

% === Package start ===
\usepackage{fontspec} % 加這個就可以設定字體
\usepackage{xeCJK} % 讓中英文字體分開設置
\usepackage{graphicx} % 引入圖片
\usepackage{indentfirst}
% === Package end ===

\setCJKmainfont{標楷體}
\XeTeXlinebreaklocale "zh" % Defines how to break lines for multilingual text
\XeTeXlinebreakskip = 0pt plus 1pt % Inter-character linebreak stretch
\graphicspath{{images/}} 
\setlength{\parindent}{2em}

% === Content start from here ===
\begin{document}

\begin{titlepage}
    \linespread{2} %兩倍行距
    \centering
    \Large 國立台灣海洋大學資訊工程學系專題報告\\
    \LARGE dARt \\
    \ \\
    \includegraphics[width=3cm]{NTOU-school-badge}\\
    \ \\
    \linespread{1}
    \begin{table}[h]
        \large
        \centering
        \begin{tabular}{lll}
            00XXXXXX CSE 5A 李 萱 \\
            00XXXXXX CSE 4A 羅 捷 \\
            00XXXXXX CSE 4A 洪晟洋 \\
            00XXXXXX CSE 4A 魏資碩 \\
        \end{tabular}
    \end{table}

    \large 報告編號: NTOUCSE 110學年度-指導老師編號-競賽組第 X 組 \\
    \large 指導教授:張欽圳老師\ \\
    \vspace{3cm}
    \large 中華民國 110年 XX 月 XX 日

\end{titlepage}

\tableofcontents % Compile twice to correctly show

% === Chapter ===
\chapter{介紹}
\section{研究目的與動機}

    研究目的

\section{系統簡介(遊戲說明、流程圖、多人運作)}

    系統簡介(遊戲說明、流程圖、多人運作)

\section{系統架構}

    系統架構

% === Chapter ===
\chapter{前端架構}
\section{技術介紹}

    技術介紹

\section{技術說明與實作}
\subsection{OpenGL 物件位置與實景貼合}

    OpenGL 物件位置與實景貼合

\subsection{ARCore 姿態估計、Sensor 校正}

    ARCore 姿態估計、Sensor 校正

% === Chapter ===
\chapter{後端架構}
\section{技術介紹}

    技術介紹

\section{技術說明與實作}
\subsection{技術說明}

    技術說明

\subsection{網路端實作}

    網路端實作

\subsection{神經網路手勢辨識}

    神經網路手勢辨識

\end{document}
